\section{Frontline}

همانطور که اشاره کردیم Frontline یک \lr{Web API} است که با زبان Go توسعه داده شده.
این سرویس درون vm در حال اجراست و از طریق پردازه فرزند کامپایلر زبان مورد نظر را فراخوانی می‌کند.

این سرویس از طریق openrc که سرویس init برای لینوکس می‌باشد بلافاصله بعد از بوت vm در حال اجرا قرار می‌گیرد.

درون فایل سیستم هر vm تمام کامپایلرهای مورد نیاز قرار دارد.
این فایل سیستم با دستور dd با فضای یک گیگابایت ساخته شده است
و از طریق docker روی آن نوشته می‌شود.

این پروسه توسط اسکریپتی انجام می‌شود. در این اسکریپت توسط alpine و پکیج منجر آن کامپایلر زبان های مختلف دانلود می‌شود.
فایل سیستم به عنوان volume برای docker اضافه می‌شود و از این طریق امکان نصب این کامپایلر ها روی آن امکان پذیر می‌شود.

هر vm نیاز به گرفتن ip دارد. VMVisor از طریق درخواست HTTP با Frontline در ارتباط است.
برای بحث نتورکینگ از CNI استفاده شده است که از بحث این مقاله خارج است.

درون Frontline دو مسیر اصلی وجود دارد.
health و exec. health برای بررسی آمادگی vm به کار می‌رود.
پیش تر اشاره کردیم که وظیفه VMVisor آماده نگه داشتن ۱۰ vm می‌باشد.
ساختن vm جدید ممکن است چندین ثانیه طول بکشید. هر ۱۰ ثانیه VMVisor درخواستی به health می‌زند و پس از دریافت ۲۰۰ آن vm را به مجموعه اضافه می‌کند.

