\section{VMVisor}
شاید پیچیده ترین سرویس این پروژه VMVisor باشد.
این سرویس چندین وظیفه دارد شامل:

\begin{itemize}
    \item مدیریت و ساختن مجموعه vm
    \item دریافت درخواست اجرا کد از صف
    \item ارتباط با Frontline برای اجرای کد
    \item ارسال تغییر وضعیت کد به صف
\end{itemize}

تعداد مجموعه vm های آماده ۱۰ عدد است. به هر vm یک عدد هسته پردازنده و ۲۵۶ مگابایت حافظه RAM و ۱ گیگابایت فضا اختصاص می‌شود.

ماکسیمم مدت زمان اجرا ۱۰ ثانیه است. پس از آن vm حذف می‌شود.
پس از اجرا یک vm به مجموعه اضافه می‌شود تا تعداد vm های آماده همان ۱۰ عدد بماند.

ارتباط VMVisor با vm از طریق سرویس به نام Frontline است.
Frontline در اصل یک \lr{REST API} است که درون vm در حال اجراست و درخواست را دریافت کرده و خروجی را به VMVisor بر می‌گرداند.
این سرویس سپس وضعیت اجرا را روی صف قرار می‌دهد.


در شکل زیر وضعیت های مختلف اجرا کد را می‌توانید ببینید.

\begin{figure}[htbp]
    \centering
    \caption{وضعیت اجرا کد}
    \label{fig:job-state}
    \begin{tikzpicture}[node distance=2cm]
        \node (received) [startstop] {Received};
        \node (running) [process, right of=received, xshift=2cm] {Running};
        \node (done) [process, right of=running, xshift=2cm] {Done};
        \node (failed_received) [process, below of=received] {Failed};
        \node (failed_running) [process, below of=running] {Failed};

        \draw [arrow] (received) -- (running);
        \draw [arrow] (running) -- (done);
        \draw [arrow] (received) -- (failed_received);
        \draw [arrow] (running) -- (failed_running);
    \end{tikzpicture}
\end{figure}

در ادامه به آخرین تکه پازل یعنی Frontline می‌پردازیم.
