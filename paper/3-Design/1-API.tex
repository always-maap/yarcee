
\section{API}

همانطور که اشاره کردم بخش API پروژه از زبان Go و فریمورک Fiber استفاده می‌کند.
این پروژه لایه ورودی ما به بخش های داخلی سیستم است.

در عکس زیر نمایی از api های موجود در پروژه مشاهده می‌شود.
این api ها در چند دسته مختلف تقسیم بندی شده اند. sandbox برای ساخت یک پروژه جدید و اجرا آن است.
بخش auth برای ثبت نام و ورود کاربر است.
بخش user برای دریافت اطلاعات کاربر وارد شده است.
و health برای بررسی liveness و readiness سیستم در نظر گرفته شده است.
وجود این مسیر باعث می‌شود در سیستم های مدیریت کانتینر مانند kubernetes از آمادگی سرویس اطمینان حاصل کرد.

\begin{figure}[htbp]
    \centering
    \includegraphics[width=1\textwidth]{./3-Design/swagger.png}
    \caption{لیست API}
    \label{fig:swagger}
\end{figure}

همچنین این سرویس به دو صف متصل است که به آن گوش می‌دهد و روی آن ارسال می‌کند.
کلاینت پس از ساخت پروژه و ویرایش کد، درخواست اجرا آن را ثبت  می‌کند. در پشت صحنه درخواستی به مسیر execute
زده می‌شود. این درخواست روی صف \lr{sandbox queue} ارسال می‌شود.

از طرفی سرویس VMVisor این درخواست را به یک vm می‌سپارد و خروجی کد و وضعیت اجرای آن را روی صف \lr{sandbox status queue} قرار می‌دهد.
سرویس API به آن صف گوش می‌دهد و آن را روی پایگاه داده می‌نویسد.
کلاینت هر از ۱۰۰ میلی ثانیه در حال درخواست برای stdout و stderr است.
