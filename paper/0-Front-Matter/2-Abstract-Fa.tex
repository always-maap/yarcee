\begin{ParsaAbstractParsi}
    محتویات چکیده شامل سه بخش مقدمه¬ای کوتاه و بیان مسأله، روش تحقیق و ویژگی¬های آن و نهایتا نتایج تحقیق و کاربردها (در صورت لزوم) باشد. در متن چکیده، از ارجاع به منابع و اشاره به جدول‌ها و نمودارها اجتناب شود. در صورت نیاز به معرفی حوزه تحقیق و مبانی تئوری آن، حداکثر در پاراگراف اول چکیده مطرح گردد. فقط به ارائهی روش تحقیق و نتایج نهایی و محوری بسنده کرده و از ارائهی موضوعات و نتایج كلی اجتناب شود. كلمات یا عباراتی كه در این بخش توضیح داده می‌شود، باید كاملاً محوری و مرتبط با موضوع تحقیق باشند. متن چکیده حداکثر یک صفحه باشد. جمله اول چکیده بیان‌کننده هدف نهایی از انجام پژوهش است. جملات بعدی به منظور توضیح پیرامون هدف اصلی آورده می‌شوند. سپس، روش‌هایی که برای انجام پروژه مورد استفاده قرار خواهند گرفت به طور خلاصه و کلی مورد اشاره قرار می‌گیرند. در نهایت، نتایج مورد انتظار، نوآوری‌ها و کاربردهای این پژوهش عنوان می‌گردند. در چكیده از ارجاع به منابع، ذكر روابط ریاضی و بیان تاریخچه خودداری ‌می‌شود.
    از زیرنویس کلمات مخفف شده در چکیده خودداری شود. کلمات مخفف شده در لیست کلمات اختصاری قابل مشاهده است.


    \bigskip\noindent
    \متن‌سیاه{کلمات کلیدی:} تعداد كلمات یا عبارات كلیدی حداكثر می‌تواند هفت كلمه یا عبارت باشد که با نقطه ویرگول (؛) از یک‌دیگر جدا شده‌اند. مانند: سختی؛ تنش؛ عملکرد
\end{ParsaAbstractParsi}
\clearpage